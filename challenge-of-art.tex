\documentclass[letterpaper]{article}
\usepackage{xltxtra,verse,realscripts}
\setmainfont[Mapping=tex-text,Ligatures={Common},
             Numbers={Lining}]{Adobe Garamond Pro}
\setlength{\parskip}{1em}

\title{\textsc{The Challenge of Art}}
\author{\emph{Christopher T. Haley}}
\date{}

\begin{document}

\maketitle

Talk to many a contemporary artist---especially an aspiring
artist---read a review, listen to some professor in a MFA program,
and you'll quickly get the idea that the point of art is somehow ``to
challenge people.'' About what? Apparently anything goes. I go to the
theater, and someone is challenging my views on race or gender, in the
galleries someone is challenging my views on religion, and God knows
what they're trying to challenge in the symphony halls---perhaps just
must my patience?

Now, don't misunderstand me: I have the highest regard for art, and
for artists as well{\ldots}when they are good. But it is an objective
fact (outside of academia and the art world) that there is not a lot
of good art today. And I submit to you that the dearth of good art in
our culture has something to do with this bizarre notion of
challenging.

I don't have space here to get into a lengthy discussion on the
history of the role of the artist vis-a-vis the audience, James Joyce,
Schoenberg, and all that. But it will suffice to recount a
conversation I was fortunate to have recently with a young, talented,
and sincere Christian playwright. In discussing her new
project---about which I am personally excited---I revealed my lack of
sophistication by asking the very unartistic question: ``What's the
point?'' Of course, she was ready with an answer: to challenge x, y,
and z. But when I asked her why on earth I would pay good money to go
and have my views challenged by a playwright---well, she hadn't
thought of that. And people wonder why the arts are suffering! The art
schools teach students to challenge the audience; but no one has
taught the audience to appreciate it. Many critics even decry this
fact, blaming the poor state of the arts in our country on an audience
that just doesn't ``get it'' ---whatever ``it'' is. And indeed, we do
not get it. We do not want to get it.

The notion that the artist's duty is to challenge the audience is
offensive to the audience. It is arrogant and condescending. Learning
how to paint, sculpt, write, or compose, does not make one a moral
authority on art or anything else. There is no moral value in being
transgressive for the sake of transgressiveness. And there is no merit
in challenging people just for the sake of a challenge. The old
``devil's argument'' is, after all, a very poor argument.

It is noteworthy that this aim of the contemporary artist is absent
from most great art. Whatever the point of a great works of art, it
certainly is not to challenge people. Of course, no one will dispute
that art does challenge people: the moral difficulties in Shakespeare
and Aeschylus are challenging, Hardy's war poems are challenging,
Górecki's \emph{Third Symphony} is challenging; any great work of art
demands an appropriately great response, and that is always
challenging. But the real challenge of art is not just some point of
argument---there are no shortages of those; and I don't need art for
that---the real challenge of art is something immeasurably
greater. The challenge of art is beauty. And the challenge of beauty
is truth. Truth is challenging. But it is also inviting. It is also
glorious and liberating. Truth is wondrous, not scandalous.

When trying to figure out what the real point of art is, I find it
always a good idea to consult the poetry section before the philosophy
section. Did Homer challenge? Did Pindar? Dante? Milton? They are all
challenging, to be sure. But one would be a fool to say that the
point, the telos, of any great art or artist is primarily to
challenge. What, after all, is the challenge in Mozart or Bach?

So what is the point of art, then, if not to challenge? Rilke gives
one of the finest answers; in an inscription to a book of poems, he
writes:

\begin{minipage}{\linewidth}
\settowidth{\versewidth}{Oh speak, poet, what do you do?}
\setlength{\vgap}{11em}
\begin{verse}[\versewidth]
Oh speak, poet, what do you do? \\
\vin                    ---I praise.

But the monstrosities and the murderous days, \\
how do you endure them, how do you take them? \\
\vin                    ---I praise.

But the anonymous, the nameless grays, \\
how, poet, do you still invoke them? \\
\vin                    ---I praise.

What right have you, in all displays, \\
in very mask, to be genuine? \\
\vin                    ---I praise.

And that the stillness and the turbulent sprays \\
know you like star and storm? \\
\vin                     : ---because I praise.
\end{verse}
\end{minipage}

Praise. Celebration. Despite any self-satisfaction, any arrogance, and
rebuke or condemnation, you find always in great art something to
celebrate! Art is about celebration---not desecration. And it is just
this mistake that misleads so many of today's would-be artists. Rilke
is not unaware of the moral failures of his culture; he is not blind
to the Great War around him. Rilke, and every other great artist, had
to confront the same sorts of tragedies, hypocrisies, and injustices
that today's artists confront. But there is a world of difference
between the way great artists and today's artists respond to these
problems. The artists who endure do so because they see beyond the
problems they face, they look to what is eternal, realizing that the
evils of today are here but for today. Today's artists would rather
hold up---almost celebrate---the evils. Take, for example, one of our
finest playwrights, whom I generally admire both as a playwright and a
person: Stephen Adly Guirgis. In his plays \emph{In Arabia We'd All Be
  Kings} and \emph{Our Lady of 121\textsuperscript{st} St.}, he gives
us a gruesome, heart-wrenching picture of the underbelly of our
society, of the often (willfully) unseen results of corporate and
political malfeasance, of drugs, of hope and lost hope, of dignity
shrouded in darkness; and he does this with great empathy and
humanity. But he stops there. There is no solution. There is nothing
to be praised. He gives this hell to the audience, and we are supposed
to do something about it---that is the challenge. The difference is
that in great art, when Dante gives us Hell, he also leads us through
Purgatory and into Paradise. It is precisely this which is lacking in
art today. Our artists are content to give us hell. That is their
challenge.

What we find in truly great art, however, is not the challenge of
hell, but a glimpse of heaven. The artist does not come with demands
and accusations, but comes offering praise, delight, beauty, hope,
truth! Perhaps it is because our artists lack these very qualities
that they do not offer them? Or perhaps they have merely been taught
poorly? Whatever the cause, it seems to me that the solution is the
same. The artist's vocation---like all vocations---must be understood
as a call to love and humility. This should be at the forefront of the
artist's mind: love your audience as yourself.

This is especially true for the Christian artist. When I talk with
Christian artists, I always ask: ``Where are the beatitudes in your
art?'' Now that is a challenge. Christ is always the real
challenge. We---artists and audience---are called to serve, to be
last, to carry a cross for our neighbor. We are not called to
challenge or accuse, but to love---that is the challenge! And art will
only regain its proper and necessary place in our culture when artists
begin to meet that challenge, when they no longer see themselves as
judges, but as servants.

But that will not happen unless we, the audience, also do our part: we
must also serve the artists. We must support good art when it is
found, and we must cultivate in our schools and communities the sort
of environment in which great art can flourish. Beauty is relational;
and great art is a two-way street. So perhaps the challenge of art
belongs to all of us.
\end{document}
